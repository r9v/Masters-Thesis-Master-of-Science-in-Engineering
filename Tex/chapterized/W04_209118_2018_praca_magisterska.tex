\documentclass[eng,printmode,oneside]{mgr}
\usepackage[MeX]{polski}
\usepackage[utf8]{inputenc}
\usepackage[T1]{fontenc} 
\usepackage{graphicx}
\usepackage{subfigure}
\usepackage{psfrag}
\usepackage{amsmath}
\usepackage{amsfonts}
%\usepackage{supertabular}
\usepackage{array}
\usepackage{tabularx}
\usepackage{hhline}
\usepackage{rev}
%\usepackage{framed}
\usepackage{color}
\usepackage{url}
%\usepackage[notref]{showkeys}
%\usepackage{showlabels}
\usepackage{float}
%\usepackage{tikz}
\usepackage{enumitem} 
\usepackage{graphicx}       
\usepackage{rotating}       % pakiet umożliwiający obracanie rysunków
\usepackage{subfigure}      % pakiet umożliwiający tworzenie podrysunków
\usepackage{epic}           
\usepackage{listings}       % pakiet dedykowany zrodlom programow
\usepackage{verbatim}       % pakiet dedykowany rozmaitym wydrukom tekstowym
\usepackage{amssymb}        % pakiet z rozmaitymi symbolami matematycznymi
\usepackage{amsmath}        % pakiet z rozmaitymi środowiskami matematycznymi
\usepackage[polish]{babel}  % pakiet lokalizujący dokument w języku polskim
\usepackage[OT4]{fontenc}
\usepackage[utf8]{inputenc}
\usepackage{bm}
\usepackage{gensymb}
\usepackage{booktabs}
\usepackage{epstopdf}
\usepackage{amssymb}
\usepackage[utf8]{inputenc}
\usepackage{amsmath}
\usepackage{amsfonts}
\usepackage{amssymb}
\usepackage{graphics}
\usepackage[]{algorithmic} %pseudocode
\usepackage[]{algorithm2e} %pseudocode
\usepackage{float}
\usepackage{csquotes}
\usepackage{epsfig}
%\usepackage{hyperref}
\usepackage{wrapfig} 
\reviewer{dypl}{0.2}{0.2}{0.67}
\reviewer{prof}{0.2}{0.6}{0.2}
\def\bp{\begin{review}[prof]}
\def\ep{\end{review}}
\def\bdypl{\begin{review}[dypl]}
\def\edypl{\end{review}}
\newcommand{\R}{I\!\!R} 
\newtheorem{theorem}{Twierdzenie}[section] 
\newcommand\numberthis{\addtocounter{equation}{1}\tag{\theequation}}
\newenvironment{myitemize}%
 { \begin{list}{\labelitemi}%
   {
      \setlength{\itemsep}{0mm}%
      \setlength{\topsep}{6pt}%
      \setlength{\leftmargin}{5mm}%
     \setlength{\parsep}{1mm}%
    }
  }
{\end{list}
}
\newcounter{mycount}
\newenvironment{MYenumerate}%
 {\begin{list}{\arabic{mycount}.}%
   {\usecounter{mycount}%
%      \setlength{\topsep}{12pt}%
      \setlength{\topsep}{6pt}%
      \setlength{\itemsep}{0mm}%
      \setlength{\leftmargin}{7mm}%
     \setlength{\parsep}{1mm}%
    }%
 }%
{\end{list}}
\newenvironment{todo}
{
\label{todo}
\color{red} [
}
{]
}
%%%%%%%%%chapter no new page%
\usepackage{etoolbox}

\makeatletter
\patchcmd{\chapter}{\if@openright\cleardoublepage\else\clearpage\fi}{}{}{}
\makeatother
%%%%%%%%%%%%%
\title{?}
\engtitle{?}

\author{inż. Bernard Malec}
\supervisor{Prof.\ dr hab.\ inż.\ Alicja Mazur}


\field{Automatyka i Robotyka (AIR)}
\specialisation{Robotyka (ARR)}

%\includeonly{mch0,mch1,mch2,mch3,mch4,bibl}
%\includeonly{mch3}

\begin{document}
\bibliographystyle{plabbrv} %BibTeX polski styl bibliografii 

% te ozdobniki dodaje się dopiero na końcu 
%\maketitle

\chapter{Model satelity typu free-floating}
Równania dynamiki robota free-floating z manipulatorem można zapisać jako\\
\begin{equation}
\label{wrsdad}
H\ddot{q}+C\dot{q}=
H\ddot{q}+c=
\begin{bmatrix}
    H_b  &H_{bm}\\
    H_{bm}^T  &H_m \\
\end{bmatrix}
\left(
\begin{array}{c}
\ddot{q_b}\\
\ddot{q_m}\\
\end{array}
\right)
+
\left(
\begin{array}{c}
c_b\\
c_m\\
\end{array}
\right)
=
\left(
\begin{array}{c}
0\\
u\\
\end{array}
\right).
\end{equation}\\
\section{Model we współrzędnych barycentrycznych}
Model dynamiki (\ref{wrsdad}) można zapisać także we współrzędnych barycentrycznych.\\
Położenie środka masy robota free-floating z manipulatorem można wyliczyć z definicji jako
\begin{equation}
\left(\sum\limits^w_{i=3}m_i\right)\phi_b=\sum\limits^w_{i=3}m_ir_i,
\end{equation}
gdzie $r_i$ to środek masy członu $i$ robota w podstawowym układzie współrzędnych.\\
$\phi_b$ to współrzędne barycentryczne układu, czyli
\begin{equation}
\phi_b=
\left(
\begin{array}{c}
\bar{x}\\
\bar{y}\\
\theta\\
\end{array}
\right).
\end{equation}
Model we współrzędnych barycentrycznych można wyrazić jako
\begin{equation}
\bar{H}\ddot{\phi}+\bar{C}\dot{\phi}=
\bar{H}\ddot{\phi}+\bar{c}
=
\begin{bmatrix}
    \bar{H}_b  &\bar{H}_{bm}\\
    \bar{H}_{bm}^T  &\bar{H}_m \\
\end{bmatrix}
\left(
\begin{array}{c}
\ddot{\phi_b}\\
\ddot{q_m}\\
\end{array}
\right)
+
\left(
\begin{array}{c}
\bar{c}_b\\
\bar{c}_m\\
\end{array}
\right)
=
\left(
\begin{array}{c}
0\\
u\\
\end{array}
\right),
\end{equation}
współrzędne $\phi$ mają postać
\begin{equation}
\phi=
\left(
\begin{array}{c}
\phi_b\\
q_m\\
\end{array}
\right).
\end{equation}
Macierz $\bar{H}$ i $\bar{C}$ uzyskuje się przez przeliczenie energii kinetycznej do układu współrzędnych barycentrycznych.

\chapter{Algorytm}
\begin{equation}
y=k(\phi)
\end{equation}
\begin{equation}
\dot{y}_i=\dfrac{dk_i(\phi)}{d\phi}\dot{\phi}=J_i\dot{\phi}
\end{equation}
\begin{equation}
\ddot{y}_i=\dot{\phi}^T\dfrac{d^2k_i(\phi)}{d\phi^2}\dot{\phi}+J_i\ddot{\phi}=P_i+J_i\ddot{\phi}
\end{equation}
\begin{equation}
\ddot{\phi}=\bar{H}^{-1}
\left(
\begin{array}{c}
-\bar{c}_b\\
u-\bar{c}_m\\
\end{array}
\right)
\end{equation}
\begin{equation}
\ddot{\phi}=P+J\left(\bar{H}^{-1}\left(
\begin{array}{c}
-\bar{c}_b\\
u-\bar{c}_m\\
\end{array}
\right)\right)
\end{equation}
\begin{equation}
\ddot{\phi}=
P-J\bar{H}^{-1}\left(
\begin{array}{c}
\bar{c}_b\\
\bar{c}_m\\
\end{array}
\right)+
J\bar{H}^{-1}\left(
\begin{array}{c}
0\\
u\\
\end{array}
\right)
\end{equation}
\begin{equation}
\ddot{\phi}=
F+G\left(\begin{array}{c}
0\\
u\\
\end{array}
\right)
\end{equation}
\begin{equation}
F=P-J\bar{H}^{-1}\left(
\begin{array}{c}
\bar{c}_b\\
\bar{c}_m\\
\end{array}
\right)
\end{equation}
\begin{equation}
G=J\bar{H}^{-1}
\end{equation}
\begin{equation}
u=G_2^{-1}(-F+\zeta)
\end{equation}
\begin{equation}
\ddot{y}=\zeta
\end{equation}
\end{document}
