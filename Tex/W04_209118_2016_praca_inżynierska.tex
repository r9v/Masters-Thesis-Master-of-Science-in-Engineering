\documentclass[eng,printmode]{mgr}
\usepackage[MeX]{polski}
\usepackage[utf8]{inputenc}
\usepackage[T1]{fontenc} 
\usepackage{graphicx}
\usepackage{subfigure}
\usepackage{psfrag}
\usepackage{amsmath}
\usepackage{amsfonts}
%\usepackage{supertabular}
\usepackage{array}
\usepackage{tabularx}
\usepackage{hhline}
\usepackage{rev}
%\usepackage{framed}
\usepackage{color}
\usepackage{url}
%\usepackage[notref]{showkeys}
%\usepackage{showlabels}
\usepackage{float}
\usepackage{tikz}
\usepackage{enumitem} 
\usepackage{bm}

\reviewer{dypl}{0.2}{0.2}{0.67}
\reviewer{prof}{0.2}{0.6}{0.2}
\def\bp{\begin{review}[prof]}
\def\ep{\end{review}}
\def\bdypl{\begin{review}[dypl]}
\def\edypl{\end{review}}
\newcommand{\R}{I\!\!R} 
\newtheorem{theorem}{Twierdzenie}[section] 
\newcommand\numberthis{\addtocounter{equation}{1}\tag{\theequation}}
\newenvironment{myitemize}%
 { \begin{list}{\labelitemi}%
   {
      \setlength{\itemsep}{0mm}%
      \setlength{\topsep}{6pt}%
      \setlength{\leftmargin}{5mm}%
     \setlength{\parsep}{1mm}%
    }
  }
{\end{list}
}
\newcounter{mycount}
\newenvironment{MYenumerate}%
 {\begin{list}{\arabic{mycount}.}%
   {\usecounter{mycount}%
%      \setlength{\topsep}{12pt}%
      \setlength{\topsep}{6pt}%
      \setlength{\itemsep}{0mm}%
      \setlength{\leftmargin}{7mm}%
     \setlength{\parsep}{1mm}%
    }%
 }%
{\end{list}}
\newenvironment{todo}
{
\label{todo}
\color{red} [
}
{]
}

\title{Planowanie ruchu bramkarza sportowego jako kinematyczne zadanie robotyczne}
\engtitle{Planning a goalkeeper's motion as a kinematic task of robotics}

\author{Bernard Malec}
\supervisor{Prof.\ dr hab.\ inż.\ Ignacy Dulęba}


\field{Automatyka i Robotyka (AIR)}
\specialisation{Robotyka (ARR)}

%\includeonly{mch0,mch1,mch2,mch3,mch4,bibl}
%\includeonly{mch3}

\begin{document}
\bibliographystyle{plabbrv} %BibTeX polski styl bibliografii 

% te ozdobniki dodaje się dopiero na końcu 
%\maketitle
%\tableofcontents


\chapter{Model manipulatora mobilnego}
Manipulator mobilny to platforma mobilna z zamontowanym na niej manipulatorem. Biorąc pod uwagę typ ograniczeń nałożonych kolejno na platformę i manipulator można wydzielić 4 typy manipulatorów mobilnych:
\begin{enumerate}
\item typ (\textit{h,h}) - platforma i manipulator związane są ograniczeniami holonomicznymi,
\item typ (\textit{nh,h}) - na platformę nałożone są ograniczenia nieholonomiczne, natomiast na manipulator ograniczenia holonomiczne,
\item typ (\textit{nh,nh}) - ograniczenia nałożone na platformę i manipulator są nieholonomiczne. Taki  rodzaj manipulatora mobilnego został przedstawiony w \cite{5},
\item typ (\textit{h,nh}) - na manipulator  nałożone są ograniczenia nieholonomiczne, natomiast na platformę holonomiczne.
\end{enumerate} 

W tym rozdziale zostanie przedstawiony model manipulatora mobilnego typu (\textit{nh,h}). W szczególności w rozważanym manipulatorze platformą będzie monocykl, czyli robot mobilny klasy (2,0). Za manipulator natomiast posłuży manipulator RTR. Rozważany manipulator mobilny przedstawiono na rys 1.

Przyjmy następujący wektor współrzędnych uogólnionych manipulatora mobilnego:
\begin{equation}
q=\left(
\begin{array}{c}
q_m\\
q_r\\
\end{array}
\right)\in R^{n+p},
\end{equation}
gdzie $q_m=(x,y,\theta, \phi_1, \phi_2)^T \in R^n$ - wektor współrzędnych uogólnionych platformy mobilnej, $q_r=(q_1,q_2,q_3)^T \in R^p$ - wektor współrzędnych przegubowych manipulatora. Rozmiary przestrzeni wyniosą więc $n=5$, $p=3$.\\
Do wyprowadzenia równań manipulatora mobilnego wykorzystamy mechanikę Langrange'ą.\\ Lagranżjan L jest zdefiniowany jako różnica pomiędzy energią kinetyczną, a potencjalną układu
\begin{equation}
L(q,\dot{q})=T(q,\dot{q})-V(q).
\end{equation}
Równania ruchu przy braku sił działających na system są następujące
\begin{equation}
\label{lagrange}
\frac{d}{dt}(\frac{\partial L}{\partial\dot{q}})-\frac{\partial L}{\partial q}=0.
\end{equation}
Lewą stronę równania \eqref{lagrange} można przedstawić jako
\begin{equation}
\label{lagrange2}
\frac{d}{dt}(\frac{\partial L}{\partial \dot{q}})-\frac{\partial L}{\partial q}=Q(q)\ddot{q}+C(q,\dot{q})\dot{q}+D(q).
\end{equation}
Przy obecności sił zewnętrznych równanie \eqref{lagrange} wygląda następująco
\begin{equation}
\label{lagrange3}
Q(q)\ddot{q}+C(q,\dot{q})\dot{q}+D(q)=F_e.
\end{equation}
W naszym przypadku siły zewnętrze $F_e$ są sumą 2 składowych: $F_e=B(q)u+F_{nh}$, gdzie $B(q)u$ to uogólnione siły wejściowe(tj. wywierane na system przez elementy wykonawcze), a $F_{nh}$ to siły więzów nieholonomicznych(tj. siły zapewniające spełnienie ograniczeń nieholonomicznych).\\
\section{Ograniczenia nieholonomiczne dla platformy}
W naszym przypadku platforma manipulatora mobilnego to monocykl, który w założeniu będzie poruszać się bez poślizgu wzdłużnego i poprzecznego kół.\\
Warunek na brak poślizgu poprzecznego to\\
\begin{equation}
\dot{x}\sin(\theta)-\dot{y}\cos(\theta)=0,
\end{equation} 
a warunki na brak poślizgu wzdłużnego pierwszego i drugiego koła to:\\
\begin{equation}
\dot{x}\cos(\theta)+\dot{y}\sin(\theta)-L\dot{\theta}-R\phi_1=0,
\end{equation}
\begin{equation}
\dot{x}\cos(\theta)+\dot{y}\sin(\theta)+L\dot{\theta}-R\phi_2=0.
\end{equation}
Ograniczenia te są nieholonomiczne i można je przedstawić w postaci Pfaffa
\begin{equation}
\label{pfaff}
A(q)\dot{q}=0,
\end{equation}
gdzie
\begin{equation}
A(q)=\begin{bmatrix}
    \sin(\theta)  &-\cos(\theta) & 0&0&0&0&0&0 \\
    \cos(\theta)  &\sin(\theta) & -L&-R&0&0&0&0\\
     \cos(\theta)  &\sin(\theta) & L&0&-R&0&0&0       
\end{bmatrix}.
\end{equation}
Z \cite{5} wiemy, że siła więzów nieholonomicznych równa jest $F_{nh}=A(q)^T\lambda$, gdzie $\lambda$ to wektor mnożników Lagrange'a. Równanie (\ref{lagrange3}) można więc teraz zapisać jako
\begin{equation}
\label{lagrange4}
Q(q)\ddot{q}+C(q,\dot{q})\dot{q}+D(q)=B(q)u+A(q)^T\lambda .
\end{equation}
\section{Model w prędkościach pomocniczych}
Model postaci (\ref{lagrange4}) posiada pewne wady: macierz $B$ jest prostokątna i nie daje się odwrócić. W jawnej postaci występują również mnożniki Lagrange'a odpowiadające siłom tarcia statycznego zależnego od $u$, $q$, $\dot{q}$, $t$. Aby pozbyć się tych wad można przekształcić (\ref{lagrange4}) do modelu wyrażonego w prędkościach pomocniczych.\\
\par Najpierw zauważmy, że ograniczenia (\ref{pfaff}) wymagają aby prędkości $\dot{q}$ należały do jądra macierzy $A(q)$. Niech baza jądra $A(q)$ będzie zbiorem wektorów $\{g_1(q), g_2(q),..., g_m(q)\}$. Jeśli więc zapiszemy prędkość $\dot{q}$ w następujący sposób 
\begin{equation}
\label{niewiem1}
\dot{q}=\sum_{k=1}^{m} g_k(q)\eta_k=\left(
\begin{array}{c} g_1(q), g_2(q),..., g_m(q)\end{array}\right)\left(
\begin{array}{c}
\eta_1\\
\vdots\\
\eta_m
\end{array}\right)
=G(q)\eta
\end{equation}
to będzie ona spełniać ograniczenia (\ref{pfaff}).
Wektor $\eta$ nazywamy wektorem prędkości pomocniczych.\\
\par Zauważmy, że
\begin{equation}
\label{AG}
A(q)G(q)=0.
\end{equation}
Biorąc transpozycje (\ref{AG}) otrzymamy
\begin{equation}
\label{AG2}
G^T(q)A^T(q)=0.
\end{equation}
Aby wyeliminować mnożniki Lagrange'a z (\ref{lagrange4}) należy pomnożyć (\ref{lagrange4}) lewostronnie przez  $G^T$(dla czytelności pomińmy argumenty) i skorzystać z (\ref{AG2}). Otrzymamy wtedy
\begin{equation}
\label{predkoscipomoc1}
G^TQ\ddot{q}+G^TC\dot{q}+G^TD=G^TBu+G^TA^T\lambda =G^TBu.
\end{equation}
Zróżniczkujmy jeszcze (\ref{niewiem1}) po czasie
\begin{equation}
\label{niewiem2}
\ddot{q}=\dot{G}\eta+G\dot{\eta}.
\end{equation}
Po podstawieniu (\ref{niewiem1}) i (\ref{niewiem2}) do (\ref{predkoscipomoc1}), otrzymamy model w prędkościach pomocniczych:
\begin{equation}
G^TQ(\dot{G}\eta+G\dot{\eta})+G^TCG\eta+G^TD=G^TQG\dot{\eta}+G^T(Q\dot{G}+CG)\eta+G^TD=G^TBu,
\end{equation}
czyli
\begin{equation}
Q^*\dot{\eta}+C^*\eta+D^*=B^*u,
\end{equation}
gdzie:
\begin{align*}
 Q^*&=G^TQG,\\
 C^*&=G^T(Q\dot{G}+CG),\\
 D^*&=G^TD,\\
 B^*&=G^TB.
\end{align*}

\begin{thebibliography}{99}
\bibitem{1} Mazur A. \textit{New approach to designing input-output decoupling
controllers for mobile manipulators}. Bulettin of the Polish Academy of Science vol. 53, no. 1 (2005)

\bibitem{2} Yamamoto Y, Yun X. \textit{Coordinating locomotion and manipulation of a mobile manipulator.} IEEE Trans. on Automatic Control, vol. 39, no. 6 (1994)

\bibitem{3} Yamamoto Y, Yun X. \textit{Effect of the dynamic interaction on coordinated control of mobile manipulators.} IEEE Trans. on Robotics and Automation, vol. 12, no. 5 (1996)

\bibitem{4} Mazur A, Arent K. \textit{Lecture Notes in Control and Information Sciences, vol. 335, strony 55-71.} IEEE Trans. on Robotics and Automation, vol. 12, no. 5 (2007)

\bibitem{5} Tchoń K, Jakubiak J. \textit{Acceleration-driven kinematics of mobile manipulators: an endogenous configuration space approach w: J. Lenarcic, C. Galletti (Eds.) On Advances in Robot Kinematics.} Kluwer Academic Publishers, The Netherlands (2004)

\bibitem{5} M. R. Flannery. \textit{The Enigma of Nonholonomic Constraints.} American Association of Physics Teachers, 73(3):265–272, 2005.
\end{thebibliography}

%\addcontentsline{toc}{chapter}{Bibliografia}
\begin{thebibliography}{10}
\bibitem{AlSa16} A. Salwach, "Wizualizacja i analiza robotyczna postury bramkarza sportowego", inżynierska praca dyplomowa, Wydział
  Elektroniki, Politechnika Wrocławska 2016.  
\bibitem{Kinect} http://www.xbox.com/en-US/xbox-one/accessories/kinect
\bibitem{game} J. D. Williams, "Strateg doskonały. Wprowadzenie do teorii gier", PWN, Warszawa 1965
\bibitem{game2} http://www.fuw.edu.pl/\textasciitilde kostecki/teoria\underline{\hspace{.1in}}gier.pdf
\bibitem{MSpong} M. Spong, M. Vidyasagar, "Dynamika i sterowanie robotów", WNT, Warszawa 1997
\bibitem{Eigen} http://eigen.tuxfamily.org/
\bibitem{rapidXml} http://rapidxml.sourceforge.net/
\bibitem{doxygen} http://www.stack.nl/\textasciitilde dimitri/doxygen/
\bibitem{qt} https://www.qt.io/qt5-7/
\bibitem{body} R. Contini, "Body Segment Parametsrs, Part II", \textit{Artificial Limbs}, Washington D.C. 1972
\bibitem{Wolfram} C. Hastings, K. Mischo, M. Morrison, "Hands-On Start to Wolfram Mathematica", Wolfram Media, Inc. 2015
\bibitem{multiresgrid} R. D. Jovanović, M. Tuba, D. Simian, "An Algorithm for
  Multi-Resolution Grid Creation Applied to Explicit Finite Difference Scheme",
  Proc.\  of the 12th WSEAS Int.\ Conf.\ on COMPUTERS 2008. 
\end{thebibliography}




%\bibliography{bibliografia} % bibliografia.bib

%opcjonalnie może się tu pojawić spis rysunków i tabel
% \listoffigures
% \listoftables
\end{document}
